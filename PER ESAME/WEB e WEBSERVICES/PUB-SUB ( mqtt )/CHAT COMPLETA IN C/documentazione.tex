\documentclass[12pt,a4paper]{article}
\usepackage[utf8]{inputenc}
\usepackage[italian]{babel}
\usepackage{geometry}
\usepackage{graphicx}
\usepackage{listings}
\usepackage{xcolor}
\usepackage{hyperref}
\usepackage{fancyhdr}
\usepackage{titlesec}
\usepackage{enumitem}
\usepackage{booktabs}
\usepackage{array}
\usepackage{longtable}
\usepackage{tcolorbox}
\usepackage{amsmath}
\usepackage{amssymb}

% Impostazioni pagina
\geometry{
    top=2.5cm,
    bottom=2.5cm,
    left=2.5cm,
    right=2.5cm
}

% Stile intestazioni
\pagestyle{fancy}
\fancyhf{}
\fancyhead[L]{Chat MQTT Universitaria}
\fancyhead[R]{\thepage}
\renewcommand{\headrulewidth}{0.4pt}

% Configurazione colori
\definecolor{codeblue}{RGB}{42,63,80}
\definecolor{codegray}{RGB}{128,128,128}
\definecolor{codegreen}{RGB}{0,128,0}
\definecolor{codepurple}{RGB}{128,0,128}
\definecolor{backcolour}{RGB}{245,245,245}
\definecolor{infocolor}{RGB}{116,185,255}
\definecolor{warningcolor}{RGB}{253,203,110}

% Stile per il codice
\lstdefinestyle{mystyle}{
    backgroundcolor=\color{backcolour},
    commentstyle=\color{codegreen},
    keywordstyle=\color{magenta},
    numberstyle=\tiny\color{codegray},
    stringstyle=\color{codepurple},
    basicstyle=\ttfamily\footnotesize,
    breakatwhitespace=false,
    breaklines=true,
    captionpos=b,
    keepspaces=true,
    numbers=left,
    numbersep=5pt,
    showspaces=false,
    showstringspaces=false,
    showtabs=false,
    tabsize=2,
    frame=single,
    rulecolor=\color{black!30}
}

\lstset{style=mystyle}

% Box per informazioni e warning
\newtcolorbox{infobox}{
    colback=infocolor!10,
    colframe=infocolor!80!black,
    leftrule=4mm,
    rightrule=0mm,
    toprule=0mm,
    bottomrule=0mm,
    title=\textbf{Informazione},
    fonttitle=\bfseries
}

\newtcolorbox{warningbox}{
    colback=warningcolor!10,
    colframe=warningcolor!80!black,
    leftrule=4mm,
    rightrule=0mm,
    toprule=0mm,
    bottomrule=0mm,
    title=\textbf{Attenzione},
    fonttitle=\bfseries
}

% Titolo personalizzato
\titleformat{\section}{\Large\bfseries\color{codeblue}}{\thesection}{1em}{}
\titleformat{\subsection}{\large\bfseries\color{codeblue}}{\thesubsection}{1em}{}
\titleformat{\subsubsection}{\normalsize\bfseries\color{codeblue}}{\thesubsubsection}{1em}{}

\begin{document}

% Pagina del titolo
\begin{titlepage}
    \centering
    \vspace*{2cm}
    
    {\Huge\bfseries Chat MQTT Universitaria}
        
    \vspace{2cm}
    {\large Implementazione Paradigma PUB/SUB con Mosquitto in C}
    
    \vspace{3cm}
    \vspace{1cm}
    {\Large Mattioli Simone}
    \vspace{3cm}
    
    \begin{tcolorbox}[colback=blue!5!white,colframe=blue!75!black]
        \centering
        \textbf{Caratteristiche Principali:}
        \begin{itemize}[leftmargin=2cm]
            \item Sistema di messaging MQTT per ambiente universitario
            \item Gerarchia di ruoli con permessi differenziati
            \item Implementazione in C con libreria Mosquitto
            \item Analisi traffico con Wireshark
            \item Build automation con Makefile
        \end{itemize}
    \end{tcolorbox}
    
    \vfill
    {\large \today}
\end{titlepage}

\newpage
\tableofcontents
\newpage

\section{Introduzione}

\begin{infobox}
\textbf{Cos'è MQTT?}

MQTT (Message Queuing Telemetry Transport) è un protocollo di messaggistica leggero progettato per dispositivi con risorse limitate e reti con larghezza di banda ridotta. Utilizza il paradigma publish/subscribe per la comunicazione asincrona.
\end{infobox}

Questo progetto implementa un sistema di chat universitaria basato su MQTT che simula la comunicazione tra diversi ruoli accademici: Rettore, Segreteria, Docenti e Studenti. Il sistema utilizza una gerarchia di topic strutturata per gestire i permessi di accesso e la distribuzione dei messaggi.

\subsection{Obiettivi del Progetto}

\begin{itemize}
    \item Implementare il paradigma PUB/SUB usando Mosquitto
    \item Creare una gerarchia di permessi basata sui ruoli
    \item Dimostrare la gestione delle connessioni e dei messaggi MQTT
    \item Fornire un'interfaccia utente interattiva
    \item Permettere l'analisi del traffico con Wireshark
\end{itemize}

\section{Architettura del Sistema}

\subsection{Gerarchia dei Topic}

Il sistema utilizza una struttura gerarchica di topic MQTT per gestire i permessi e il routing dei messaggi:

\begin{lstlisting}[language=bash, caption=Struttura Topic MQTT]
universita/
├── personale/
│   ├── docenti/
│   │   └── [username]
│   ├── studenti/
│   │   └── [username]
│   └── segreteria/
│       └── [username]
└── amministrazione/
    └── rettore/
        └── [username]
\end{lstlisting}

\subsection{Ruoli e Permessi}

\subsubsection{RETTORE}
\begin{itemize}
    \item \textbf{Sottoscrizione:} \texttt{universita/\#} (accesso completo)
    \item \textbf{Pubblicazione:} \texttt{universita/amministrazione/rettore/[username]}
    \item \textbf{Privilegi:} Può leggere tutti i messaggi dell'università
\end{itemize}

\subsubsection{SEGRETERIA}
\begin{itemize}
    \item \textbf{Sottoscrizione:} \texttt{universita/personale/\#}
    \item \textbf{Pubblicazione:} \texttt{universita/personale/segreteria/[username]}
    \item \textbf{Privilegi:} Accesso a tutto il personale (docenti e studenti)
\end{itemize}

\subsubsection{DOCENTE}
\begin{itemize}
    \item \textbf{Sottoscrizione:} \texttt{universita/personale/\#}
    \item \textbf{Pubblicazione:} \texttt{universita/personale/docenti/[username]}
    \item \textbf{Privilegi:} Può comunicare con altri docenti e studenti
\end{itemize}

\subsubsection{STUDENTE}
\begin{itemize}
    \item \textbf{Sottoscrizione:} \texttt{universita/personale/studenti/\#}
    \item \textbf{Pubblicazione:} \texttt{universita/personale/studenti/[username]}
    \item \textbf{Privilegi:} Può comunicare solo con altri studenti
\end{itemize}

\subsection{Flusso dei Messaggi}

\begin{longtable}{|p{3cm}|p{5cm}|p{6cm}|}
\hline
\textbf{Mittente} & \textbf{Destinatari} & \textbf{Topic} \\
\hline
Studente & Altri Studenti, Docenti, Segreteria, Rettore & universita/personale/studenti/[username] \\
\hline
Docente & Altri Docenti, Segreteria, Rettore & universita/personale/docenti/[username] \\
\hline
Segreteria & Altre Segreterie, Rettore & universita/personale/segreteria/[username] \\
\hline
Rettore & Tutti & universita/amministrazione/rettore/[username] \\
\hline
\end{longtable}

\section{Installazione e Configurazione}

\subsection{Prerequisiti}

\begin{itemize}
    \item macOS con Homebrew installato
    \item Compilatore GCC
    \item Accesso terminale
\end{itemize}

\subsection{Installazione Dipendenze}

\begin{lstlisting}[language=bash, caption=Installazione Mosquitto]
# Installa Mosquitto broker e librerie di sviluppo
brew install mosquitto
brew install mosquitto-dev

# Verifica installazione
mosquitto --help
mosquitto_pub --help
mosquitto_sub --help
\end{lstlisting}

\subsection{Struttura del Progetto}

\begin{lstlisting}[language=bash, caption=Struttura Directory]
mqtt-chat/
├── mqtt_chat.c          # Codice sorgente principale
├── Makefile            # Build automation
└── README.md           # Documentazione base
\end{lstlisting}

\subsection{Compilazione}

\begin{lstlisting}[language=bash, caption=Comandi di Build]
# Compilazione standard
make

# Compilazione con debug
make debug

# Pulizia file compilati
make clean

# Installazione dipendenze
make install-deps
\end{lstlisting}

\section{Guida all'Utilizzo}

\subsection{Avvio del Sistema}

\subsubsection{1. Avviare il Broker Mosquitto}
\begin{lstlisting}[language=bash]
# Avvio del servizio
make start-broker
# oppure
brew services start mosquitto

# Verifica stato
make status-broker
# oppure  
brew services list | grep mosquitto
\end{lstlisting}

\subsubsection{2. Avviare l'Applicazione}
\begin{lstlisting}[language=bash]
./mqtt_chat
\end{lstlisting}

\subsection{Interfaccia Utente}

\subsubsection{Selezione Ruolo}
All'avvio, il programma richiede di scegliere il ruolo:

\begin{lstlisting}[language=bash]
=== SCEGLI IL TUO RUOLO ===
1. Rettore (legge tutto)
2. Segreteria (legge personale)  
3. Docente (legge personale)
4. Studente (legge solo studenti)
Scelta [1-4]: 
\end{lstlisting}

\subsubsection{Comandi Disponibili}

\begin{longtable}{|p{3cm}|p{11cm}|}
\hline
\textbf{Comando} & \textbf{Descrizione} \\
\hline
\texttt{/help} & Mostra l'elenco dei comandi disponibili \\
\hline
\texttt{/quit} & Esce dall'applicazione \\
\hline
\texttt{/status} & Mostra lo stato della connessione e il ruolo attivo \\
\hline
\texttt{/topic} & Visualizza i topic di sottoscrizione e pubblicazione \\
\hline
\texttt{[messaggio]} & Qualsiasi altro testo viene inviato come messaggio \\
\hline
\end{longtable}

\subsection{Scenario di Test}

Per replicare uno scenario completo, aprire 6 terminali:

\begin{lstlisting}[language=bash, caption=Setup Multi-Utente]
# Terminal 1 - Rettore
./mqtt_chat
→ Scegli ruolo: 1
→ Username: rettore

# Terminal 2 - Segreteria 1  
./mqtt_chat
→ Scegli ruolo: 2
→ Username: segretaria1

# Terminal 3 - Segreteria 2
./mqtt_chat  
→ Scegli ruolo: 2
→ Username: segretaria2

# Terminal 4 - Docente
./mqtt_chat
→ Scegli ruolo: 3  
→ Username: prof_rossi

# Terminal 5 - Studente 1
./mqtt_chat
→ Scegli ruolo: 4
→ Username: mario_rossi

# Terminal 6 - Studente 2  
./mqtt_chat
→ Scegli ruolo: 4
→ Username: giulia_bianchi
\end{lstlisting}

\section{Analisi del Codice Sorgente}

\subsection{Strutture Dati Principali}

Il file \texttt{mqtt\_chat.c} definisce le seguenti strutture fondamentali:

\subsubsection{Enumerazione Ruoli}
\begin{lstlisting}[language=c, caption=Definizione Ruoli Utente]
typedef enum {
    ROLE_RETTORE,
    ROLE_SEGRETERIA,
    ROLE_DOCENTE,
    ROLE_STUDENTE
} user_role_t;
\end{lstlisting}

\subsubsection{Struttura Client Chat}
\begin{lstlisting}[language=c, caption=Struttura Client MQTT]
typedef struct {
    struct mosquitto *mosq;              // Istanza client Mosquitto
    user_role_t role;                    // Ruolo dell'utente
    char username[MAX_USERNAME_LEN];     // Nome utente
    char subscribe_topic[MAX_TOPIC_LEN]; // Topic sottoscrizione
    char publish_base_topic[MAX_TOPIC_LEN]; // Topic pubblicazione base
    int is_connected;                    // Stato connessione
} mqtt_chat_client_t;
\end{lstlisting}

\subsection{Funzioni Callback}

\subsubsection{Callback di Connessione}
\begin{lstlisting}[language=c, caption=Gestione Connessione]
void on_connect(struct mosquitto *mosq, void *userdata, int result) {
    mqtt_chat_client_t *client = (mqtt_chat_client_t *)userdata;
    
    if (result == 0) {
        printf("✅ Connesso al broker MQTT come %s\n", client->username);
        client->is_connected = 1;
        
        // Subscribe al topic appropriato
        mosquitto_subscribe(mosq, NULL, client->subscribe_topic, 0);
        printf("📡 Sottoscritto al topic: %s\n", client->subscribe_topic);
    } else {
        printf("❌ Errore di connessione: %d\n", result);
    }
}
\end{lstlisting}

Questa funzione:
\begin{itemize}
    \item Gestisce il risultato della connessione al broker
    \item Effettua automaticamente la sottoscrizione al topic appropriato
    \item Aggiorna lo stato di connessione del client
    \item Fornisce feedback visivo all'utente
\end{itemize}

\subsubsection{Callback di Messaggio}
\begin{lstlisting}[language=c, caption=Ricezione Messaggi]
void on_message(struct mosquitto *mosq, void *userdata, 
                const struct mosquitto_message *message) {
    mqtt_chat_client_t *client = (mqtt_chat_client_t *)userdata;
    
    if (message->payloadlen) {
        // Ottieni timestamp
        time_t now = time(NULL);
        struct tm *tm_info = localtime(&now);
        char timestamp[20];
        strftime(timestamp, sizeof(timestamp), "%H:%M:%S", tm_info);
        
        // Stampa il messaggio ricevuto
        printf("\n💬 [%s] Topic: %s\n", timestamp, message->topic);
        printf("   Messaggio: %s\n", (char *)message->payload);
        printf(">> ");
        fflush(stdout);
    }
}
\end{lstlisting}

Funzionalità implementate:
\begin{itemize}
    \item Riceve e processa i messaggi in arrivo
    \item Formatta l'output con timestamp e informazioni topic
    \item Gestisce la visualizzazione asincrona dei messaggi
    \item Mantiene il prompt utente sempre visibile
\end{itemize}

\subsection{Inizializzazione Client}

\begin{lstlisting}[language=c, caption=Configurazione Ruolo]
void init_client_by_role(mqtt_chat_client_t *client, user_role_t role, 
                        const char *username) {
    client->role = role;
    strncpy(client->username, username, MAX_USERNAME_LEN - 1);
    client->is_connected = 0;
    
    switch (role) {
        case ROLE_RETTORE:
            strcpy(client->subscribe_topic, "universita/#");
            strcpy(client->publish_base_topic, "universita/amministrazione/rettore");
            break;
            
        case ROLE_SEGRETERIA:
            strcpy(client->subscribe_topic, "universita/personale/#");
            strcpy(client->publish_base_topic, "universita/personale/segreteria");
            break;
            
        case ROLE_DOCENTE:
            strcpy(client->subscribe_topic, "universita/personale/#");
            strcpy(client->publish_base_topic, "universita/personale/docenti");
            break;
            
        case ROLE_STUDENTE:
            strcpy(client->subscribe_topic, "universita/personale/studenti/#");
            strcpy(client->publish_base_topic, "universita/personale/studenti");
            break;
    }
}
\end{lstlisting}

Questa funzione implementa la logica di controllo accesso configurando i topic appropriati per ogni ruolo secondo la gerarchia definita.

\subsection{Funzione di Pubblicazione}

\begin{lstlisting}[language=c, caption=Invio Messaggi]
void publish_message(mqtt_chat_client_t *client, const char *message) {
    if (!client->is_connected) {
        printf("❌ Non connesso al broker!\n");
        return;
    }
    
    char full_topic[MAX_TOPIC_LEN];
    char full_message[MAX_MESSAGE_LEN];
    
    // Crea il topic completo con l'username
    snprintf(full_topic, sizeof(full_topic), "%s/%s", 
             client->publish_base_topic, client->username);
    
    // Crea il messaggio completo con nome utente
    snprintf(full_message, sizeof(full_message), "%s: %s", 
             client->username, message);
    
    int result = mosquitto_publish(client->mosq, NULL, full_topic, 
                                  strlen(full_message), full_message, 0, false);
    
    if (result == MOSQ_ERR_SUCCESS) {
        printf("📤 Messaggio inviato!\n");
    } else {
        printf("❌ Errore invio messaggio: %d\n", result);
    }
}
\end{lstlisting}

Caratteristiche implementative:
\begin{itemize}
    \item Verifica stato connessione prima dell'invio
    \item Costruisce topic completo includendo username
    \item Formatta il messaggio con identificazione mittente
    \item Gestisce errori di pubblicazione con feedback utente
    \item Utilizza QoS 0 per massime performance
\end{itemize}

\subsection{Loop Principale}

Il \texttt{main()} implementa un ciclo di gestione input che:
\begin{itemize}
    \item Inizializza la libreria Mosquitto
    \item Gestisce selezione ruolo e username
    \item Configura callback e connessione
    \item Avvia loop network in background
    \item Processa comandi utente in loop sincrono
    \item Gestisce cleanup e disconnessione pulita
\end{itemize}

\section{Analisi del Makefile}

Il Makefile fornisce un sistema di build automation completo:

\subsection{Variabili di Configurazione}

\begin{lstlisting}[language=make, caption=Configurazione Build]
CC = gcc
CFLAGS = -Wall -Wextra -std=c99 -pedantic
LIBS = -lmosquitto
TARGET = mqtt_chat
SOURCE = mqtt_chat.c
\end{lstlisting}

\subsection{Target Principali}

\subsubsection{Compilazione}
\begin{lstlisting}[language=make]
$(TARGET): $(SOURCE)
	$(CC) $(CFLAGS) -o $(TARGET) $(SOURCE) $(LIBS)

debug: CFLAGS += -g -DDEBUG
debug: $(TARGET)
\end{lstlisting}

\subsubsection{Gestione Dipendenze}
\begin{lstlisting}[language=make]
install-deps:
	@echo "Installazione dipendenze per macOS..."
	brew install mosquitto
	brew install mosquitto-dev
\end{lstlisting}

\subsubsection{Gestione Broker}
\begin{lstlisting}[language=make]
start-broker:
	@echo "Avvio broker Mosquitto..."
	brew services start mosquitto

stop-broker:
	@echo "Arresto broker Mosquitto..."
	brew services stop mosquitto

status-broker:
	@echo "Stato broker Mosquitto:"
	brew services list | grep mosquitto
\end{lstlisting}

Il Makefile automatizza completamente il workflow di sviluppo, dalla gestione delle dipendenze al controllo del broker MQTT.

\section{Testing e Debug}

\subsection{Analisi con Wireshark}

\subsubsection{Filtri Utili}

\begin{longtable}{|p{6cm}|p{8cm}|}
\hline
\textbf{Filtro} & \textbf{Descrizione} \\
\hline
\texttt{mqtt} & Mostra tutto il traffico MQTT \\
\hline
\texttt{mqtt.topic contains "chat"} & Filtra per topic contenenti "chat" \\
\hline
\texttt{mqtt and ip.addr == localhost} & Traffico MQTT verso/da localhost \\
\hline
\texttt{mqtt.msgtype == 3} & Solo messaggi PUBLISH \\
\hline
\end{longtable}

\subsubsection{Osservazioni sul Traffico}
\begin{itemize}
    \item \textbf{Keep Alive:} Pacchetti PING regolari per mantenere connessione NAT
    \item \textbf{Publisher Pattern:} Connessione → Pubblicazione → Disconnessione
    \item \textbf{Subscriber Pattern:} Connessione persistente con keep-alive
    \item \textbf{QoS Levels:} Il sistema usa QoS 0 (at most once delivery)
\end{itemize}

\subsection{Debug e Logging}

\subsubsection{Compilazione Debug}
\begin{lstlisting}[language=bash]
make debug
\end{lstlisting}

\subsubsection{Debugging con GDB}
\begin{lstlisting}[language=bash]
gdb ./mqtt_chat
(gdb) run
(gdb) break on_message
(gdb) continue
\end{lstlisting}

\subsection{Test di Funzionalità}

\subsubsection{Test di Connessione}
\begin{enumerate}
    \item Avviare il broker Mosquitto
    \item Lanciare il client
    \item Verificare messaggio "Connesso al broker MQTT"
    \item Controllare sottoscrizione automatica
\end{enumerate}

\subsubsection{Test di Comunicazione}
\begin{enumerate}
    \item Avviare due istanze con ruoli compatibili
    \item Inviare messaggio dalla prima istanza
    \item Verificare ricezione nella seconda istanza
    \item Controllare timestamp e formattazione
\end{enumerate}

\section{Considerazioni sulla Sicurezza}

\subsection{Implementazioni di Sicurezza Attuali}
\begin{itemize}
    \item \textbf{Controllo Accesso Logico:} Basato su topic hierarchy
    \item \textbf{Isolamento Ruoli:} Ogni ruolo ha permessi specifici
    \item \textbf{Validazione Input:} Controllo lunghezza messaggi e topic
    \item \textbf{Gestione Errori:} Handling sicuro delle connessioni
\end{itemize}

\subsection{Vulnerabilità Identificate}

\begin{warningbox}
\textbf{Messaggi in Chiaro}

I messaggi viaggiano non crittografati. Un attaccante con accesso alla rete può intercettare e leggere tutti i contenuti.

\textbf{Mancanza di Autenticazione}

Non c'è verifica dell'identità degli utenti. Chiunque può assumere qualsiasi ruolo.

\textbf{Topic Spoofing}

Un client può potenzialmente pubblicare su topic non autorizzati modificando il codice.
\end{warningbox}

\subsection{Miglioramenti Suggeriti}
\begin{enumerate}
    \item \textbf{TLS/SSL:} Crittografia del trasporto
    \item \textbf{Autenticazione:} Username/password o certificati
    \item \textbf{ACL Broker:} Controllo accesso server-side
    \item \textbf{Message Encryption:} Crittografia end-to-end
    \item \textbf{Rate Limiting:} Prevenzione spam/DoS
    \item \textbf{Audit Logging:} Log delle attività utente
\end{enumerate}

\section{Risoluzione Problemi}

\subsection{Problemi Comuni}

\subsubsection{Errore: "Broker non raggiungibile"}
\begin{lstlisting}[language=bash]
# Verifica se Mosquitto è attivo
brew services list | grep mosquitto

# Se non attivo, avvialo
brew services start mosquitto

# Testa connessione manuale
mosquitto_sub -h localhost -t test/topic
\end{lstlisting}

\subsubsection{Errore di Compilazione: "mosquitto.h not found"}
\begin{lstlisting}[language=bash]
# Installa librerie di sviluppo
brew install mosquitto-dev

# Verifica path include
pkg-config --cflags libmosquitto
\end{lstlisting}

\subsubsection{Messaggi Non Ricevuti}
\begin{enumerate}
    \item Verifica topic di sottoscrizione con \texttt{/topic}
    \item Controlla che i ruoli abbiano permessi compatibili
    \item Testa con client da linea di comando:
    \begin{lstlisting}[language=bash]
mosquitto_sub -h localhost -t "universita/#"
    \end{lstlisting}
\end{enumerate}

\subsection{Diagnostica Avanzata}

\subsubsection{Log del Broker}
\begin{lstlisting}[language=bash]
# Avvia broker in modalità verbose
mosquitto -v

# Log su file  
mosquitto -v > mosquitto.log 2>&1
\end{lstlisting}

\subsubsection{Monitor Traffico di Rete}
\begin{lstlisting}[language=bash]
# Con netstat
netstat -an | grep 1883

# Con lsof
lsof -i :1883
\end{lstlisting}

\begin{infobox}
\textbf{Suggerimento per il Debug}

Usa sempre Wireshark per analizzare il traffico MQTT. È il modo più efficace per capire cosa succede a livello di protocollo e identificare problemi di comunicazione.
\end{infobox}

\section{Possibili Estensioni}

\subsection{Funzionalità Avanzate}
\begin{itemize}
    \item \textbf{Interfaccia Grafica:} GUI con GTK+ o Qt
    \item \textbf{Persistenza Messaggi:} Database SQLite per storico chat
    \item \textbf{Notifiche Push:} Integrazione con sistemi di notifica OS
    \item \textbf{File Transfer:} Invio di allegati tramite MQTT
    \item \textbf{Presenza Utenti:} Indicator online/offline
    \item \textbf{Gruppi Privati:} Chat room tematiche
\end{itemize}

\subsection{Integrazione Enterprise}
\begin{itemize}
    \item \textbf{LDAP Integration:} Autenticazione aziendale
    \item \textbf{SSO Support:} Single Sign-On
    \item \textbf{Compliance:} GDPR, logging audit
    \item \textbf{Scalabilità:} Clustering broker, load balancing
\end{itemize}

\section{Conclusioni}

Il progetto Chat MQTT Universitaria rappresenta un'implementazione completa e funzionale del paradigma publish/subscribe utilizzando il protocollo MQTT. L'architettura basata sulla gerarchia di topic permette un controllo granulare dei permessi di accesso, simulando realisticamente l'ambiente comunicativo di un'istituzione universitaria.

\subsection{Obiettivi Raggiunti}
\begin{itemize}
    \item Implementazione completa del pattern PUB/SUB
    \item Sistema di ruoli e permessi funzionante
    \item Interfaccia utente intuitiva e interattiva
    \item Gestione robusta delle connessioni e disconnessioni
    \item Analisi del traffico di rete tramite Wireshark integrata
    \item Build system automatizzato con Makefile completo
\end{itemize}

\subsection{Lezioni Apprese}
\begin{itemize}
    \item \textbf{Paradigma PUB/SUB:} Efficacia nella disaccoppiamento dei componenti
    \item \textbf{Topic Hierarchy:} Importanza cruciale nella progettazione dei permessi
    \item \textbf{Gestione Asincrona:} Complessità nella sincronizzazione I/O e UI
    \item \textbf{Error Handling:} Necessità di gestione robusta degli stati di errore
    \item \textbf{Debugging Distribuito:} Utilità fondamentale degli strumenti di analisi di rete
\end{itemize}

\subsection{Limitazioni Identificate}
\begin{itemize}
    \item Mancanza di persistenza dei messaggi offline
    \item Assenza di crittografia end-to-end
    \item Controllo accesso basato solo su logica client-side
    \item Scalabilità limitata per grandi numeri di utenti
    \item Gestione basilare degli errori di rete
\end{itemize}

\subsection{Applicabilità Reale}
Il sistema implementato fornisce una base solida per applicazioni IoT e sistemi di messaging distribuiti. Le competenze acquisite sono direttamente trasferibili a:

\begin{itemize}
    \item Sistemi di monitoraggio industriale
    \item Applicazioni IoT smart home
    \item Piattaforme di telemetria veicolare
    \item Sistemi di notifica enterprise
    \item Architetture microservizi event-driven
\end{itemize}

\section{Appendici}

\subsection{Appendice A: Codice Completo}

Il codice sorgente completo è disponibile nei file del progetto. Le sezioni principali implementano:

\begin{itemize}
    \item Inizializzazione e configurazione client MQTT
    \item Gestione callback per eventi di rete
    \item Loop principale con input utente non-bloccante
    \item Sistema di comandi interattivi
    \item Gestione sicura della memoria e cleanup
\end{itemize}

\subsection{Appendice B: Configurazioni Avanzate}

\subsubsection{Configurazione Broker Mosquitto}
Per ambienti di produzione, modificare il file \texttt{/opt/homebrew/etc/mosquitto/mosquitto.conf}:

\begin{lstlisting}[language=bash, caption=mosquitto.conf Avanzato]
# Porta di ascolto
port 1883

# Autenticazione
allow_anonymous false
password_file /opt/homebrew/etc/mosquitto/passwd

# ACL (Access Control List)
acl_file /opt/homebrew/etc/mosquitto/acl

# Persistenza
persistence true
persistence_location /opt/homebrew/var/lib/mosquitto/

# Log
log_dest file /opt/homebrew/var/log/mosquitto/mosquitto.log
log_type all

# Security
max_connections 1000
max_inflight_messages 20
max_queued_messages 1000
\end{lstlisting}

\subsubsection{Setup ACL di Produzione}
\begin{lstlisting}[language=bash, caption=File ACL Esempio]
# ACL per ruoli universitari
user rettore
topic readwrite universita/#

user segreteria_#
topic readwrite universita/personale/#

user docente_#
topic readwrite universita/personale/docenti/#
topic read universita/personale/studenti/#

user studente_#
topic readwrite universita/personale/studenti/#
\end{lstlisting}

\subsection{Appendice C: Script di Deployment}

\begin{lstlisting}[language=bash, caption=Deploy Script]
#!/bin/bash
# deploy.sh - Script di deployment automatico

set -e

echo "=== Deploy Chat MQTT Universitaria ==="

# Verifica prerequisiti
check_dependencies() {
    echo "Verifica dipendenze..."
    command -v gcc >/dev/null 2>&1 || { echo "GCC richiesto"; exit 1; }
    command -v mosquitto >/dev/null 2>&1 || { echo "Mosquitto richiesto"; exit 1; }
    echo "✅ Dipendenze verificate"
}

# Compilazione
build_project() {
    echo "Compilazione progetto..."
    make clean
    make
    echo "✅ Compilazione completata"
}

# Setup servizi
setup_services() {
    echo "Configurazione servizi..."
    brew services start mosquitto
    echo "✅ Broker MQTT avviato"
}

# Test sistema
run_tests() {
    echo "Esecuzione test..."
    # Test connessione broker
    timeout 5 mosquitto_pub -h localhost -t test -m "test" || {
        echo "❌ Test broker fallito"
        exit 1
    }
    echo "✅ Test completati"
}

# Main
main() {
    check_dependencies
    build_project
    setup_services
    run_tests
    echo "🚀 Deploy completato con successo!"
}

main "$@"
\end{lstlisting}

\subsection{Appendice D: Metriche e Monitoring}

\subsubsection{Script di Monitoraggio}
\begin{lstlisting}[language=bash, caption=Monitoring Script]
#!/bin/bash
# monitor.sh - Monitoraggio sistema MQTT

while true; do
    echo "=== MQTT System Status $(date) ==="
    
    # Status broker
    echo "Broker Status:"
    brew services list | grep mosquitto
    
    # Connessioni attive
    echo "Active Connections:"
    netstat -an | grep 1883 | wc -l
    
    # Utilizzo risorse
    echo "Resource Usage:"
    ps aux | grep mosquitto | grep -v grep
    
    sleep 30
done
\end{lstlisting}

\section{Bibliografia e Riferimenti}

\subsection{Documentazione Tecnica}
\begin{itemize}
    \item Eclipse Mosquitto Documentation: \texttt{https://mosquitto.org/documentation/}
    \item MQTT Protocol Specification v5.0: \texttt{https://docs.oasis-open.org/mqtt/mqtt/v5.0/}
    \item Eclipse Paho C Client Library: \texttt{https://github.com/eclipse/paho.mqtt.c}
    \item Wireshark MQTT Dissector Guide: \texttt{https://wiki.wireshark.org/MQTT}
\end{itemize}

\subsection{Risorse Aggiuntive}
\begin{itemize}
    \item HiveMQ MQTT Essentials: \texttt{https://www.hivemq.com/mqtt-essentials/}
    \item MQTT Security Best Practices: \texttt{https://www.hivemq.com/blog/mqtt-security-fundamentals/}
    \item IoT Communication Protocols Comparison
    \item Real-time Messaging Patterns in Distributed Systems
\end{itemize}

\subsection{Standard e RFC}
\begin{itemize}
    \item RFC 6455 - The WebSocket Protocol
    \item RFC 7693 - The BLAKE2 Cryptographic Hash and Message Authentication Code
    \item ISO/IEC 20922:2016 - Information technology -- Message Queuing Telemetry Transport (MQTT)
\end{itemize}

\begin{infobox}
\textbf{Nota Finale}

Questa documentazione rappresenta una guida completa per l'implementazione, il deployment e la manutenzione del sistema Chat MQTT Universitaria. Per aggiornamenti e contributi al progetto, consultare il repository Git associato.
\end{infobox}

\vfill
\centering
\textit{Fine Documentazione} \\
\textbf{Chat MQTT Universitaria v1.0}

\end{document}