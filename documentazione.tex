\documentclass[12pt,a4paper]{article}
\usepackage[utf8]{inputenc}
\usepackage[italian]{babel}
\usepackage{amsmath}
\usepackage{amsfonts}
\usepackage{amssymb}
\usepackage{graphicx}
\usepackage{hyperref}
\usepackage{listings}
\usepackage{xcolor}
\usepackage{array}
\usepackage{booktabs}
\usepackage{geometry}
\usepackage{fancyhdr}
\usepackage{tcolorbox}

% Configurazione pagina
\geometry{margin=2.5cm}
\pagestyle{fancy}
\fancyhf{}
\fancyhead[L]{Documentazione Repository PSR}
\fancyhead[R]{Simone Mattioli}
\fancyfoot[C]{\thepage}

% Configurazione colori e stili
\definecolor{codegreen}{rgb}{0,0.6,0}
\definecolor{codegray}{rgb}{0.5,0.5,0.5}
\definecolor{codepurple}{rgb}{0.58,0,0.82}
\definecolor{backcolour}{rgb}{0.95,0.95,0.92}

\lstdefinestyle{mystyle}{
    backgroundcolor=\color{backcolour},
    commentstyle=\color{codegreen},
    keywordstyle=\color{magenta},
    numberstyle=\tiny\color{codegray},
    stringstyle=\color{codepurple},
    basicstyle=\ttfamily\footnotesize,
    breakatwhitespace=false,
    breaklines=true,
    captionpos=b,
    keepspaces=true,
    numbers=left,
    numbersep=5pt,
    showspaces=false,
    showstringspaces=false,
    showtabs=false,
    tabsize=2
}

\lstset{style=mystyle}

\hypersetup{
    colorlinks=true,
    linkcolor=blue,
    filecolor=magenta,
    urlcolor=cyan,
    pdftitle={Documentazione Repository Programmazione e Sicurezza delle Reti},
    pdfauthor={Simone Mattioli},
    bookmarksopen=true,
}

\title{
    \vspace{-2cm}
    \begin{tcolorbox}[colback=blue!5!white,colframe=blue!75!black,title=Documentazione Della Repository GitHub]
        \Huge \textbf{Corso di Programmazione e\\Sicurezza delle Reti}\\\\
        \vspace{0.5cm}
        \normalsize Repository di Simone Mattioli
    \end{tcolorbox}
}

\author{
    Università degli Studi di Verona\\
    \vspace{1cm}
    \textbf{Docente:} Prof. Davide Quaglia\\
    \textbf{Studente:} Dott. Mattioli Simone\\
    \vspace{1cm}
    \href{https://github.com/simo-hue/Corso-Programmazione-Sicurezza-delle-Reti}{\texttt{GitHub Repository}}
}

\date{Anno Accademico 2024/2025}

\begin{document}

\maketitle
\newpage

\tableofcontents
\newpage

\section{Introduzione}

Questo documento fornisce una documentazione completa del repository GitHub \textbf{"Corso-Programmazione-Sicurezza-delle-Reti"} sviluppato da Simone Mattioli per il corso di Programmazione e Sicurezza delle Reti presso l'Università degli Studi di Verona, tenuto dal Prof. Davide Quaglia.

Il repository rappresenta una collezione organizzata di materiali didattici, codice sorgente, appunti e risorse pratiche sviluppate durante il corso nell'anno accademico 2024/2025.

\subsection{URL del Repository}
\url{https://github.com/simo-hue/Corso-Programmazione-Sicurezza-delle-Reti}

\section{Obiettivi del Repository}

Il progetto è stato creato con i seguenti obiettivi principali:

\begin{itemize}
    \item \textbf{Esempi Pratici:} Fornire implementazioni concrete di programmazione di rete utilizzando diversi linguaggi (C, Java, HTML/JavaScript)
    \item \textbf{Documentazione di Laboratorio:} Documentare procedure pratiche per l'utilizzo di strumenti come Wireshark, MQTT, WebSocket e REST
    \item \textbf{Materiale di Studio:} Conservare materiali didattici, appunti personali e risorse per la preparazione all'esame
    \item \textbf{Riferimento Completo:} Creare un punto di riferimento organizzato per tutti gli argomenti del corso
\end{itemize}

\section{Requisiti di Sistema}

Per utilizzare completamente i materiali presenti nel repository, sono necessari i seguenti software:

\begin{table}[h!]
\centering
\begin{tabular}{|l|l|p{6cm}|}
\hline
\textbf{Software} & \textbf{Versione Minima} & \textbf{Note} \\
\hline
GCC / Clang & 9.0 & Su Windows è consigliato utilizzare Cygwin o WSL \\
\hline
Make & -- & Alcuni esempi includono Makefile per la compilazione automatica \\
\hline
Wireshark & 4.x & Strumento fondamentale per l'analisi del traffico di rete \\
\hline
Mosquitto & 2.x & Necessario solamente per le esercitazioni MQTT \\
\hline
Python 3 & 3.10 & Per gli script di supporto opzionali \\
\hline
\end{tabular}
\caption{Requisiti software minimi}
\end{table}

\begin{tcolorbox}[colback=yellow!10!white,colframe=orange!75!black,title=Nota Importante]
Per istruzioni dettagliate sull'installazione dell'ambiente C su Linux, macOS e Windows, consultare il file \texttt{Impostazione-PC-applicazioni-socket.pdf} presente nel repository.
\end{tcolorbox}

\section{Struttura del Repository}

Il repository è organizzato in cartelle tematiche per facilitare la navigazione e l'utilizzo dei materiali:

\begin{table}[h!]
\centering
\begin{tabular}{|l|p{8cm}|}
\hline
\textbf{Cartella} & \textbf{Descrizione} \\
\hline
\texttt{DOMANDE ESAME/} & Raccolta di domande degli esami orali delle sessioni precedenti \\
\hline
\texttt{FILE FORNITI DAL PROF/} & File originali forniti dal docente tramite la piattaforma Moodle del corso (senza soluzioni) \\
\hline
\texttt{MIEI ESERCIZI/} & Soluzioni personali sviluppate dall'autore per gli esercizi di laboratorio \\
\hline
\texttt{PER ESAME/} & Soluzioni commentate e materiale ottimizzato per la preparazione all'esame \\
\hline
\texttt{APPUNTI GOODNOTES/} & Appunti manoscritti in formato PDF esportati da GoodNotes (in arrivo) \\
\hline
\texttt{LUCIDI/} & Slide ufficiali del corso in formato PDF \\
\hline
\texttt{.vscode/} & Configurazioni e task per Visual Studio Code \\
\hline
\end{tabular}
\caption{Struttura delle cartelle del repository}
\end{table}

\section{Guida Rapida all'Utilizzo}

\subsection{Compilazione degli Esempi}

Per compilare ed eseguire i principali esempi di programmazione di rete:

\subsubsection{Esempi UDP}
\begin{lstlisting}[language=bash, caption=Compilazione server/client UDP]
# Compilazione del server UDP
gcc network.c serverUDP.c -o serverUDP -lpthread

# Compilazione del client UDP
gcc network.c clientUDP.c -o clientUDP -lpthread
\end{lstlisting}

\subsubsection{Esempi TCP}
\begin{lstlisting}[language=bash, caption=Compilazione server/client TCP]
# Compilazione del server TCP
gcc network.c serverTCP.c -o serverTCP -lpthread

# Compilazione del client TCP
gcc network.c clientTCP.c -o clientTCP -lpthread
\end{lstlisting}

\subsection{Esecuzione e Test}

\subsubsection{Test Socket UDP/TCP}
\begin{enumerate}
    \item Aprire due terminali separati
    \item Nel primo terminale, compilare ed eseguire il server:
    \begin{lstlisting}[language=bash]
./serverUDP  # oppure ./serverTCP
    \end{lstlisting}
    \item Nel secondo terminale, compilare ed eseguire il client:
    \begin{lstlisting}[language=bash]
./clientUDP  # oppure ./clientTCP
    \end{lstlisting}
    \item Utilizzare Wireshark per filtrare per porta o protocollo e osservare i pacchetti scambiati
\end{enumerate}

\subsubsection{Server HTTP}
\begin{lstlisting}[language=bash, caption=Avvio server HTTP]
# Compilazione
gcc serverHTTP.c -o serverHTTP -lpthread

# Esecuzione sulla porta 8000
./serverHTTP 8000

# Apertura nel browser
xdg-open "http://127.0.0.1:8000/"
\end{lstlisting}

\subsubsection{WebSocket}
Per testare le comunicazioni WebSocket:
\begin{enumerate}
    \item Avviare il server (file \texttt{websocket\_server.c} o equivalente)
    \item Aprire il file \texttt{client.html} in un browser moderno
    \item Utilizzare l'interfaccia per inviare messaggi broadcast a tutti i client connessi
\end{enumerate}

\subsubsection{API REST}
\begin{lstlisting}[language=bash, caption=Test API REST]
# Compilazione e avvio del server REST
gcc serverHTTP-REST.c -o rest_server -lpthread
./rest_server 8080

# Test con curl
curl "http://127.0.0.1:8080/api/somma?x=5&y=7"
\end{lstlisting}

\subsubsection{MQTT}
\begin{lstlisting}[language=bash, caption=Test protocollo MQTT]
# Avvio del broker Mosquitto
mosquitto -v

# Esecuzione del publisher
./publisher

# Esecuzione del subscriber
./subscriber
\end{lstlisting}

\section{Analisi del Traffico con Wireshark}

Il repository include file di cattura (\texttt{.pcapng}) per l'analisi del traffico di rete. Per utilizzarli efficacemente:

\begin{enumerate}
    \item Aprire il file \texttt{.pcapng} corrispondente all'esercizio
    \item Utilizzare la funzione "Follow TCP/UDP Stream" per ricostruire il dialogo completo
    \item Analizzare i numeri di sequenza, i flag, l'handshake e le eventuali ritrasmissioni
    \item Confrontare il comportamento osservato con quello teorico atteso
\end{enumerate}

\section{Documentazione Teorica}

\subsection{Slide del Corso}
Nella cartella \texttt{LUCIDI/} sono disponibili le slide ufficiali del corso che coprono i seguenti argomenti:
\begin{itemize}
    \item Programmazione socket
    \item Protocollo HTTP
    \item Tecnologie WebSocket
    \item Architetture REST
    \item Protocollo MQTT
    \item Architettura TCP/IP
\end{itemize}

\subsection{Appunti Personali}
Gli appunti personali dell'autore sono disponibili in \texttt{APPUNTI GOODNOTES/}, mentre il file \texttt{2\_ripasso-reti.pdf} fornisce un riassunto rapido e conciso dei concetti fondamentali di rete in vista dell'esame.

\section{Roadmap e Sviluppi Futuri}

\subsection{TODO List}
Il repository è in continua evoluzione. Gli sviluppi futuri pianificati includono:

\begin{itemize}
    \item \textbf{Automazione Build:} Aggiunta di \texttt{Makefile} e script CI con GitHub Actions
    \item \textbf{Test di Integrazione:} Automatizzazione dei test di integrazione per i servizi REST
    \item \textbf{Documentazione:} Trascrizione degli appunti GoodNotes in formato Markdown per una migliore accessibilità
    \item \textbf{Espansione Contenuti:} Aggiunta di ulteriori esempi pratici e casi d'uso avanzati
\end{itemize}

\section{Contributi e Collaborazione}

\subsection{Linee Guida per i Contributi}
Il repository accoglie contributi dalla comunità. Per mantenere la coerenza del progetto:

\begin{enumerate}
    \item Utilizzare branch descrittivi (es. \texttt{feature/nuova-funzionalità}, \texttt{fix/correzione-bug})
    \item Seguire la convenzione di stile \texttt{clang-format} fornita nel file \texttt{.clang-format} (work-in-progress)
    \item Aggiornare la documentazione quando si modificano o aggiungono funzionalità
    \item Testare accuratamente le modifiche prima di sottomettere una pull-request
\end{enumerate}

\subsection{Processo di Contribuzione}
\begin{enumerate}
    \item Fork del repository
    \item Creazione di un branch per la propria modifica
    \item Implementazione delle modifiche
    \item Test delle funzionalità
    \item Sottomissione di una pull-request con descrizione dettagliata
\end{enumerate}

\section{Licenza e Utilizzo}

\begin{tcolorbox}[colback=red!10!white,colframe=red!75!black,title=Importante - Licenza]
Il materiale è fornito esclusivamente per uso didattico. La licenza è attualmente in fase di definizione. È necessario contattare l'autore prima di ridistribuire pubblicamente il contenuto del repository.
\end{tcolorbox}

\section{Crediti e Riconoscimenti}

\subsection{Autore Principale}
\textbf{Simone Mattioli}
\begin{itemize}
    \item Studente di Informatica, Università degli Studi di Verona
    \item Sviluppatore del repository
    \item Anno di corso: 2024/2025
\end{itemize}

\subsection{Docente del Corso}
\textbf{Prof. Davide Quaglia}
\begin{itemize}
    \item Docente del corso di Programmazione e Sicurezza delle Reti
    \item Università degli Studi di Verona
    \item Supervisione accademica del progetto
\end{itemize}

\section{Link Utili e Risorse Aggiuntive}

\subsection{Repository e Codice}
\begin{itemize}
    \item \href{https://github.com/simo-hue/Corso-Programmazione-Sicurezza-delle-Reti}{Repository GitHub Principale}
    \item \href{https://github.com/simo-hue}{Profilo GitHub dell'autore}
    \item \href{https://www.linkedin.com/in/simonemattioli2003/}{Linkedin dell'autore}
\end{itemize}

\subsection{Strumenti e Software}
\begin{itemize}
    \item \href{https://www.wireshark.org/}{Wireshark - Analizzatore di protocolli di rete}
    \item \href{https://mosquitto.org/}{Eclipse Mosquitto - Broker MQTT}
    \item \href{https://gcc.gnu.org/}{GCC - GNU Compiler Collection}
    \item \href{https://code.visualstudio.com/}{Visual Studio Code}
\end{itemize}

\subsection{Documentazione di Riferimento}
\begin{itemize}
    \item RFC 793 - Transmission Control Protocol
    \item RFC 768 - User Datagram Protocol  
    \item RFC 2616 - Hypertext Transfer Protocol HTTP/1.1
    \item RFC 6455 - The WebSocket Protocol
    \item RFC 3986 - Uniform Resource Identifier (URI)
\end{itemize}

\section{Conclusioni}

Il repository "Corso-Programmazione-Sicurezza-delle-Reti" rappresenta una risorsa completa e ben organizzata per lo studio della programmazione di rete e dei protocolli di sicurezza. La struttura modulare, la documentazione dettagliata e la varietà di esempi pratici lo rendono uno strumento prezioso sia per gli studenti del corso che per chiunque voglia approfondire questi argomenti.

La continua evoluzione del progetto, testimoniata dalla roadmap di sviluppi futuri e dall'apertura ai contributi della comunità, garantisce che questa risorsa rimanga aggiornata e sempre più completa nel tempo.

L'approccio pratico, combinato con una solida base teorica fornita dalle slide del corso e dagli appunti personali, offre un percorso di apprendimento completo che spazia dalla teoria all'implementazione pratica, dall'analisi del traffico di rete alla comprensione dei protocolli moderni.

\vfill
\begin{center}
\textit{Ultima revisione: \today}
\end{center}

\end{document}